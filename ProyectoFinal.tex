\documentclass{article}

%%%%
% PLOTS mapas y conglomerados
% bibliografia
%%%%
\usepackage[utf8]{inputenc}
\usepackage{longtable}
\usepackage{authblk}
\usepackage{adjustbox}

\usepackage{natbib}



\title{LOS INDICES DE COLOMBIA}
% autores
\renewcommand\Authand{, y }
\author[1]{\normalsize David Pulido Suarez}
\author[2]{\normalsize Luis Diaz Bautista}

\affil[1,2]{\small  Facultad de Ingeniería,Universidad de los Andes\\
\texttt{{ds.pulido10,lc.diaz12}@uniandes.edu.col}}
%\affil[1]{\small Instituto de altas investigaciones financieras\\
%Banco del Parque\\
%\texttt{delcurso@bp.com.col}}

\date{29 de Junio de 2018}

%%%%
\usepackage{Sweave}
\begin{document}
\Sconcordance{concordance:ProyectoFinal.tex:ProyectoFinal.Rnw:%
1 30 1 1 0 26 1 1 5 10 0 1 2 1 1 1 2 17 0 1 2 3 4 1 1 1 8 48 1}


\maketitle


\begin{abstract}
Este es mi primer trabajo en exploracion y modelamiento de indices usando LATEX. Este trabajo lo he hecho bajo la filosofía de trabajo replicable. Este es mi primer trabajo en exploracion y modelamiento de indices usando LATEX. Este trabajo lo he hecho bajo la filosofía de trabajo replicable. Este es mi primer trabajo en exploracion y modelamiento de indices usando LATEX. Este trabajo lo he hecho bajo la filosofía de trabajo replicable. Este es mi primer trabajo en exploracion y modelamiento de indices usando LATEX. Este trabajo lo he hecho bajo la filosofía de trabajo replicable.
\end{abstract}

\section*{Introducción}

Aqui les presento mi investigacion sobre diversos indices sociales en el mundo. Los indices los conseguí de wikipedia, espero que les gusten mucho. Aqui les presento mi investigacion sobre diversos indices sociales en el mundo. Los indices los conseguí de wikipedia, espero que les gusten mucho.Aqui les presento mi investigacion sobre diversos indices sociales en el mundo. Los indices los conseguí de wikipedia, espero que les gusten mucho.Aqui les presento mi investigacion sobre diversos indices sociales en el mundo. Los indices los conseguí de wikipedia, espero que les gusten mucho.
Aqui les presento mi investigacion sobre diversos indices sociales en el mundo. Los indices los conseguí de wikipedia, espero que les gusten mucho.Aqui les presento mi investigacion sobre diversos indices sociales en el mundo. Los indices los conseguí de wikipedia, espero que les gusten mucho.Aqui les presento mi investigacion sobre diversos indices sociales en el mundo. Los indices los conseguí de wikipedia, espero que les gusten mucho.

Comencemos viendo que hay en la sección %\ref{univariada} en la página \pageref{univariada}.

\clearpage



\section{Exploración Univariada}\label{univariada}

En esta sección exploro cada índice. En esta sección exploro cada índice. En esta sección exploro cada índice. En esta sección exploro cada índice. En esta sección exploro cada índice. En esta sección exploro cada índice. En esta sección exploro cada índice. En esta sección exploro cada índice. En esta sección exploro cada índice.



\begin{Schunk}
\begin{Soutput}
'data.frame':	32 obs. of  6 variables:
 $ IDH               : num  0.879 0.867 0.865 0.849 0.842 0.839 0.837 0.835 0.834 0.832 ...
 $ Departamento      : chr  "Santander" "Casanare" "Valle del Cauca" "Antioquia" ...
 $ Poblacion.Cabecera: int  1587972 281548 4169553 5262172 742812 761658 10070801 2438533 56487 506254 ...
 $ Poblacion.Resto   : int  502867 93701 586560 1428858 539251 206109 914484 107391 21926 68756 ...
 $ Poblacion.Total   : int  2090839 375249 4756113 6691030 1282063 967767 10985285 2545924 78413 575010 ...
 $ DepartamentoNorm  : chr  "Santander" "Casanare" "Valle del Cauca" "Antioquia" ...
\end{Soutput}
\end{Schunk}

veamos que pasa aqui
\begin{Schunk}
\begin{Soutput}
      IDH         Departamento       Poblacion.Cabecera Poblacion.Resto  
 Min.   :0.6910   Length:32          Min.   :   13090   Min.   :  21926  
 1st Qu.:0.7680   Class :character   1st Qu.:  234624   1st Qu.:  96969  
 Median :0.8040   Mode  :character   Median :  717197   Median : 268112  
 Mean   :0.8018                      Mean   : 1196730   Mean   : 360590  
 3rd Qu.:0.8343                      3rd Qu.:  970925   3rd Qu.: 487530  
 Max.   :0.8790                      Max.   :10070801   Max.   :1428858  
 Poblacion.Total    DepartamentoNorm  
 Min.   :   43446   Length:32         
 1st Qu.:  371161   Class :character  
 Median : 1028429   Mode  :character  
 Mean   : 1557320                     
 3rd Qu.: 1512087                     
 Max.   :10985285                     
\end{Soutput}
\end{Schunk}
histograma 1 
histograma 2
histograma 3
Para conocer el comportamiento de las variables se ha preparado la Tabla %\ref{Tfrecuencias}, donde se describe la distribución de las modalidades de cada variable. Los números representan la situación de algun país en ese indicador, donde el mayor valor numérico es la mejor situación.



\section{Exploración Bivariada}

En este trabajo estamos interesados en el impacto de los otros indices en el nivel de Democracia. Veamos las relaciones bivariadas que tiene esta variable con todas las demás:




Veamos la correlación entre las variables independientes:




Lo visto en la Tabla %\ref{corrTableX} se refuerza claramente en la Figura \ref{corrPlotX}.




\clearpage

\section{Modelos de Regresión}

Finalmente, vemos los modelos propuestos. Primero sin la libertad mundial como independiente, y luego con está. Los resultados se muestran en la Tabla %\ref{regresiones} de la página \pageref{regresiones}.





Como se vió en la Tabla %\ref{regresiones}, cuando está presente el \emph{indice de libertad mundial}, el \emph{índice de libertad de prensa} pierde significancia.

\clearpage

\section{Exploración Espacial}

Como acabamos de ver en la Tabla %\ref{regresiones} en la página \pageref{regresiones}, si quisieras sintetizar la multidimensionalidad de nuestros indicadores, podríamos usar tres de las cuatro variables que tenemos (un par de las originales tiene demasiada correlación). 

Así, propongo que calculemos conglomerados de países usando toda la información de tres de los indicadores. Como nuestras variables son ordinales utilizaremos un proceso de conglomeración donde las distancia serán calculadas usando la medida {\bf gower} propuestas en %\cite{gower_general_1971}, y para los enlazamientos usaremos la técnica de {\bf medoides} según \cite{reynolds_clustering_2006}. Los tres conglomerados se muestran en la Figura \ref{clustmap}.






%\bibliographystyle{apalike}
%\renewcommand{\refname}{Referencias}
%\bibliography{colombia}

\end{document}
